% !TeX root = RJwrapper.tex
\title{Ideas in Kane's Local Hillclimbing on an Economic Landscape}
\author{by Sohum Patnaik '18}

\maketitle

\abstract{
In this paper we carry out the ideas in Kane's
\emph{Local Hillclimbing on an Economic Landscape} (1996) by applying
optimization strategies on models outlined in his paper.
}

\subsection{Introduction}\label{introduction}

Firms have always been interested in finding the optimal allocation of
resources to produce the greatest profits. However, simulations have
often been inaccurate since proposed profit maximizing functions were
tractable and easy to maximize. This is not the case for firms in the
real world. Kane (1996) argues that profit functions that are difficult
to maximize for all agents are more worthwhile to utilize for modeling
purposes. Consequently, Kane (1996) proposes a profit function that
enables various strategies of firms to be tested on complex landscapes
that mimic a real world economy.

In this paper, we carry out Kane's (1996) ideas and analyze various
strategies for profit maximization. We see how the behavior of firms
affects those firms' long-run profits and end up with similar findings
to Kane (1996): firms that move slower in maximizing their profits
generally end up with higher long-run profits than firms that move
quickly in maximizing their profits.

\subsection{Data and Methods}\label{data-and-methods}

For our analysis, we generated a random input vector on a landscape
based off a given budget and number of inputs. We then carried out
various optimization strategies where firms search their neighborhood,
as defined by Kane (1996), to analyze how the different strategies
perform in trying to maximize long-run profits. Multiple strategies were
used with a varying number of connections, which determines how many
neighbors every point in each input vector has. For all our tables,
we had a budget of fifty and twenty inputs. All of our tables' values
are multiplied by one hundred.

\subsection{Strategies}\label{strategies}

Hillclimbing strategies and many of its variants, several of which were
demonstrated by Kane (1996), are primarily used in our analysis.

The Steepest Ascent Strategy in Tables I, II, and III is a strategy that
moves to the neighbor with the highest profit and continues to do so
until a local maximum is reached.

The Median Ascent Strategy in Tables I, II, and III is a strategy that
moves to the neighbor whose profits are the median of the neighbors with
profits higher than the current input vector and continues to do so
until a local maximum is reached.

The Least Ascent Strategy in Tables I, II, and III is a strategy that
moves to the neighbor with the lowest profit that is still higher than
the current input vector's profit and continues to do so until a local
maximum is reached.

\begin{center}
\hspace{2cm} \textbf{Number of Connections Per Input}
\begin{flushleft}
\hspace{2.5cm} \textbf{Strategy}
\end{flushleft}
\begin{Schunk}

\begin{tabular}{l|c|c|c|c|c}
\hline
  & 1 & 2 & 3 & 4 & 5\\
\hline
Steepest Ascent & 14.7 (0.2) & 24.1 (0.2) & 31.5 (0.3) & 37.6 (0.3) & 43.2 (0.4)\\
\hline
Median Ascent & 16.5 (0.2) & 25.5 (0.3) & 32.7 (0.3) & 38.2 (0.4) & 43.3 (0.4)\\
\hline
Least Ascent & 19.1 (0.2) & 29.4 (0.3) & 39.2 (0.5) & 43 (0.5) & 48.9 (0.6)\\
\hline
\end{tabular}

\end{Schunk}
\\
\emph{Table I}: The mean normalized profits for one thousand landscapes. Coefficients for the profit function come from [-1,1]. Standard errors are in parentheses.
\end{center}

In Table I, the Median Ascent Strategy consistently outperforms the Steepest Ascent
Strategy, and the Least Ascent Strategy consistently outperforms the
Median Ascent Strategy. We see that strategies that move through a
landscape with the intent of slowly increasing their profit values end
up doing better than strategies that try to increase their profit values
as quickly as possible.

\newpage

\begin{center}
\hspace{2cm} \textbf{Number of Connections Per Input}
\begin{flushleft}
\hspace{2.5cm} \textbf{Strategy}
\end{flushleft}
\begin{Schunk}

\begin{tabular}{l|c|c|c|c|c}
\hline
  & 1 & 2 & 3 & 4 & 5\\
\hline
Steepest Ascent & -7.3 (0.2) & -0.2 (0.2) & 6.4 (0.2) & 13 (0.3) & 18.5 (0.3)\\
\hline
Median Ascent & -6.6 (0.2) & 0.8 (0.2) & 8 (0.3) & 13.8 (0.3) & 19.8 (0.4)\\
\hline
Least Ascent & -2.9 (0.3) & 9.4 (0.3) & 23.9 (0.5) & 36.2 (0.7) & 42.5 (0.7)\\
\hline
\end{tabular}

\end{Schunk}
\\
\emph{Table II}: The mean normalized profits for one thousand landscapes. Coefficients for the profit function come from [0,1] for linear terms and squared terms, and from [-1,0] for cross products. Standard errors are in parentheses.
\end{center}

Even though a different profit function (due to a different range of
coefficients) is used in Table II than in Table I, similar results are
found. The Least Ascent Strategy constantly outperforms the other two
strategies that move input vectors to neighbors with higher profit
levels.

\begin{center}
\hspace{2cm} \textbf{Number of Connections Per Input}
\begin{flushleft}
\hspace{2.5cm} \textbf{Strategy}
\end{flushleft}
\begin{Schunk}

\begin{tabular}{l|c|c|c|c|c}
\hline
  & 1 & 2 & 3 & 4 & 5\\
\hline
Steepest Ascent & 11.6 (0.4) & 18.5 (0.3) & 21.7 (0.3) & 22.3 (0.3) & 23 (0.3)\\
\hline
Median Ascent & 13.2 (0.3) & 19.3 (0.3) & 22.5 (0.3) & 22.6 (0.3) & 23.5 (0.3)\\
\hline
Least Ascent & 13.9 (0.3) & 21.2 (0.3) & 22.6 (0.3) & 23.3 (0.3) & 23.1 (0.3)\\
\hline
\end{tabular}

\end{Schunk}
\\
\emph{Table III}: The mean normalized profits for one thousand landscapes. Coefficients for the profit function are 0 for linear terms, come from [-19,0] for squared terms, and come from [0,2] for cross products. Standard errors are in parentheses.
\end{center}

Table III is interesting to analyze since even though we find a similar
trend to Kane's (1996) findings, the values of the tables themselves
differ greatly. The profit function used to generate this table
penalizes input values that are different from each other. However, we
still find that the Least Ascent Strategy outperforms the other two
strategies in every case other than when there are five connections per
input.

The different versions of the profit functions shown through Tables I,
II, and III clearly demonstrate that strategies that slowly move to
neighbors with higher profit levels than the current input vector's
profit generally end up with higher long-run profits than strategies
that quickly move to neighbors with the highest possible profits.
However, it is worth comparing those results with a strategy where an
input vector moves to a random neighbor with a profit level higher than
its own to see if it will yield results that are similar to one of the
strategies used in Tables I, II, and III.

The Random Ascent Strategy in Table IV is a strategy that carries out
the Stochastic Hill Climbing strategy where an input vector moves to a
randomly chosen neighbor whose profit is higher than the current input
vector's profit level and continues to do so until a local maximum is
reached.

\begin{center}
\hspace{2cm} \textbf{Number of Connections Per Input}
\begin{flushleft}
\hspace{2.5cm} \textbf{Strategy}
\end{flushleft}
\begin{Schunk}

\begin{tabular}{l|c|c|c|c|c}
\hline
  & 1 & 2 & 3 & 4 & 5\\
\hline
Random Ascent & 16.1 (0.2) & 25.5 (0.3) & 32.6 (0.3) & 39.9 (0.4) & 45.3 (0.5)\\
\hline
\end{tabular}

\end{Schunk}
\\
\emph{Table IV}: The mean normalized profits for one thousand landscapes. Coefficients for the profit function come from [-1,1]. Standard errors are in parentheses.
\end{center}

Given that we use the same profit function in Table IV as the profit
function used in Table I, it is most appropriate to compare those two
tables. We find that the Random Ascent Strategy gives us results that
are closest to the Median Ascent Strategy. This makes sense because the
table shows us the means of the normalized profits over one thousand
landscapes, so the Random Ascent Strategy should return results that are
similar to the average of the Least Ascent Strategy and Steepest Ascent
Strategy, which is similar to the Median Ascent Strategy.

We have found that strategies that constantly move to neighbors with
higher profit levels than their own usually end up finding a local
maximum. This prevents them from finding the global maximum that most
firms are often looking for. Therefore, it would be interesting to
implement a strategy which gives firms the option of moving to a
neighbor with a lower profit level than their current input vector, as
this can possibly lead to ultimately finding an input vector with a
profit level that is closer to the global maximum.

Also, given the profit function that Kane (1996) proposes and the
results of the long-run profits found in Tables I, II, III, and IV, it
would be interesting to look at a strategy that continues moving to
neighbors until all of the budget is expended on a single point of an
input vector (usually on the point with the highest coefficient) since
those input vectors will often yield higher profits than input vectors
where the budget is distributed evenly throughout the vector.

The Budget Steepest Ascent Strategy in Table V is a strategy that moves
to the neighboring input vectors with the highest profits until a single
point of the input vector has all of the budget or until the process is
repeated one thousand times.

The Budget Median Ascent Strategy in Table V is a strategy that moves to
a neighboring input vector which has a median profit level of the
neighboring input levels that have higher profit levels than the current
input vector until a single point of the input vector has all of the
budget or until the process is repeated one thousand times.

The Budget Least Ascent Strategy in Table V is a strategy that moves to
a neighboring input vector with the lowest profit that is still higher
than the profit of the current input vector until a single point of the
input vector has all of the budget or until the process is repeated one
thousand times.

\begin{center}
\hspace{2cm} \textbf{Number of Connections Per Input}
\begin{flushleft}
\hspace{2.5cm} \textbf{Strategy}
\end{flushleft}
\begin{Schunk}

\begin{tabular}{l|c|c|c|c|c}
\hline
  & 1 & 2 & 3 & 4 & 5\\
\hline
Steepest Ascent & 62.3 (1.1) & 65.8 (1.1) & 66.6 (1.1) & -0.1 (0.2) & 0.2 (0.2)\\
\hline
Median Ascent & 62.2 (1.1) & 63.3 (1.1) & 66.6 (1) & 65.8 (1) & 62.2 (1.1)\\
\hline
Least Ascent & 62.1 (1.1) & 59.5 (1) & 56.8 (1.1) & 59.2 (1.1) & 59.3 (1)\\
\hline
\end{tabular}

\end{Schunk}
\\
\emph{Table V}: The mean normalized profits for one thousand landscapes. Coefficients for the profit function come from [-1,1]. Standard errors are in parentheses.
\end{center}

The three strategies utilized in Table V mostly yield much higher
results than the results of the strategies used in the previous tables.

All three strategies perform similarly with one connection, but the
results become more interesting as the number of connections increases.
While we have normally found the slowest-moving strategy to outperform
the other strategies, we find that in Table V, the Budget Least Ascent
Strategy, the slowest-moving strategy of the three strategies that are
used, finds lower long-run profits than the other strategies when there
are two and three connections. When there are four and five connections,
the Budget Median Ascent Strategy outperforms the Budget Least Ascent
Strategy. It is even more interesting to note that as there are four and
five connections, the long-run profits from the Budget Steepest Ascent
Strategy drops to near zero. These results can probably be attributed to
the fact that this strategy allows for the movement in directions that
would result in lower profits.

Given the strategies used, the Budget Median Ascent Strategy and the
Budget Least Ascent Strategy have been found to be safer strategies to
use than the Budget Steepest Ascent Strategy, especially as the number
of connections increases. This could be because the Budget Steepest
Ascent Strategy does not stop until a single point in the input vector
has all of the budget or until the process is repeated one thousand
times.

If there are a lot of connections, there are a lot of neighbors, which
increases the chances of the neighbor with the highest profit being in a
direction which distributes the budget throughout the input vector as
opposed to concentrating all of the budget into a single point of the
input vector. As long as no single point has all of the budget, the
strategy can move in a direction that reduces the profit level, and when
the function finally stops once it has repeated one thousand times, it
is at a profit level that is near zero.

\subsection{Conclusion}\label{conclusion}

This paper implements the ideas in Kane's paper
\emph{Local Hillclimbing on an Economic Landscape} (1996) by using the
proposed complex profit landscape to analyze different optimization
strategies. The real world economy is complex and difficult to optimize,
so employing models which are similarly complex and difficult to
optimize is only appropriate

Our findings are comparable to that of Kane (1996): strategies that
slowly increase in profit levels as they venture through a complex
economic landscape generally yield better long-run profits than
strategies that constantly work to move in directions that will yield
the greatest immediate profit levels. However, we also found that a
strategy that employs the idea of moving in a direction that may yield
lower profit levels in hopes that it prevents itself from becoming stuck
at a low local maximum of a landscape can have variable results.
Strategies similar to the ones successful in this paper can be used in
other simulations and ultimately by firms to maximize long-run profits.

\subsection{References}\label{references}

Kane, David. Local Hillclimbing on an Economic Landscape.
\emph{Evolutionary Programming V}. Santa Fe Institute, 1996.

\bibliography{patnaik2}

\address{
Sohum Patnaik\\
\\
}
\href{mailto:sp12@williams.edu}{\nolinkurl{sp12@williams.edu}}

